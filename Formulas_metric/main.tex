\documentclass[12pt, oneside, a4paper]{book}

% Font Encoding: Permite escribir más carácteres
\usepackage[utf8]{inputenc} 
\usepackage[T1]{fontenc} 

\usepackage[main=spanish, english]{babel} % Establecer idioma
\usepackage{url} % Para insertar url sin problemas
\usepackage{hyperref}

% Márgenes de espaciado de hoja
\usepackage[top=2.5cm, bottom=2.5cm, left=2.5cm, right=2.5cm]{geometry} 
% \usepackage[margin=2.5cm]{geometry} % O usando margin para los 4 espacios

% Fonts: Times se asemeja a Times New Roman 
\usepackage{times} % Use Times as default text font
\usepackage{mathptmx} % Use Times as default text font, and provide maths support
\usepackage{amsmath,amsfonts,amssymb} % Permitir usar $$ para formulas

% O tambien usando fontspec para el font Times New Roman en Latex
%\usepackage{fontspec} 
%\setmainfont{Times New Roman}% solo funciona con compilador XeLaTeX

% Modificando estilos de chapters, sections, subsectiones
\usepackage{titlesec}

\titleformat{\chapter}[block]
	{
		\vspace{0cm} % Espacio vertical antes del texto
		\large
		\bfseries
		\centering
	}
	{\thechapter.}
	{1em} %  "I." -- espacio -- "Descripcion" 
	{\vspace{0cm}} % Espacio vertical despues del texto

\titleformat{\section}[block]
	{\vspace{0cm}\bfseries}
	{\thesection.}
	{1em}
	{\vspace{0cm}}

\titleformat{\subsection}[block]
	{\vspace{0cm}\bfseries}
	{\thesubsection.}
	{1em}
	{\vspace{0cm}}

\titleformat{\subsubsection}[block]
	{\vspace{0cm}\bfseries}
	{\thesubsubsection.}
	{1em}
	{\vspace{0cm}}


% Modificando estilo de numeracion de chapters
\renewcommand{\thechapter}{\Roman{chapter}} %.
\renewcommand{\thesection}{\arabic{chapter}.\arabic{section}}


% Cambiar cuadrados por circulos en listas 
\renewcommand{\labelitemi}{\textbullet}
\setlength{\leftmargini}{1em}


% Modificar encabezados con fancyhdr (no hacerlos visibles) y asignar numeracion de pagina
\usepackage{fancyhdr}
\pagestyle{fancy}  
\fancyhf{}  
\lhead{}  
\cfoot{\thepage} % Numero de pagina centrado  
\renewcommand{\headrulewidth}{0pt} % Eliminar linea de encabezado


% Eliminar el indent al inicio del parrafo
\setlength{\parindent}{0cm}
\setlength{\parskip}{0em}


% Espacio entre lineas con factor de escala
\usepackage{setspace}
\renewcommand{\baselinestretch}{1.5}

%\AtBeginDocument{
	%\renewcommand{\contentsname}{INDICE GENERAL}
	% Cambia el nombre a un chapter en la tabla de contenidos	
%}
% ---------------------------------------------------

% Informacion inicial
\title{Formulas usadas para estimar Evapotranspiracion \\ Modelo METRIC}
\author{Cesar Francisco Vilca Gamarra, 2022}
\date{} % Omite la fecha debajo de author


\begin{document}

\maketitle % Carga la "Informacion inicial"

\frontmatter % The \frontmatter command makes the pages numbered in lowercase roman (v, vii), NO ENUMERA LOS CHAPTERS, although each chapter title appears in the table of contents

\tableofcontents

\mainmatter % The \mainmatter command changes the behavior back to the expected version, and RESETS THE PAGE NUMBER.

% Cargando archivos
\chapter{INTRODUCCIÓN}

En el presente documento se detallará los procesos necesarios para implementar el modelo METRIC (Allen et. al, 2007) en la plataforma Google Earth Engine, usando el lenguaje de programación Python.

\chapter{BALANCE DE ENERGÍA} 
\vspace{0.4cm}

\subsection{Radiación Neta}

Morse et al. 2000:

\begin{equation}
R_n = (1- \alpha)R_{S\downarrow} + (R_{L\downarrow} - R_{L\uparrow}) - (1- \epsilon_0)R_{L\downarrow}
\label{eq:1}
\end{equation}

Dónde: 
\begin{itemize}
    \item $R_n$ : Flujo de radiación neta $[W/m^2]$
    \item $\alpha$ : Albedo de superficie
    \item $R_{S\downarrow}$ : Radiación de onda corta entrante $[W/m^2]$
    \item $R_{L\downarrow}$ : Radiación de onda larga entrante $[W/m^2]$
    \item $R_{L\uparrow}$ : Radiación de onda larga saliente $[W/m^2]$
    \item $\epsilon_0$ : Emisividad del ancho de banda en la superficie / broad-band surface thermal emissivity
\end{itemize}

\vspace{0.4cm}

\subsubsection{Albedo}

Para obtener el albedo es necesario primero realizar una corrección atmosférica, dónde desaroolla la metodología propuesta por Tasumi et. al (2007).

\begin{equation}
\rho_{s,b} = \frac{R_{\text{out},s,b}}{R_{\text{in},s,b}} = \frac{\rho_{t,b} - C_b(1-\tau_{in,b})}{\tau_{in,b}.\tau_{out,b}}
\label{eq:1}
\end{equation}


Dónde:
\begin{itemize}
    \item $R_{\text{in},s,b}$ and $R_{\text{out},s,b}$: at-surface  hemispherical incoming and reflected radiances $[W m^{-2} \mu m^{-1}]$
    \item $\tau_{in,b}$ : trasmitancias de la radiacion solar de entrada / effective narrowband transmittance for incoming solar radiation.
    \item $\tau_{out,b}$ : y la de radiacion de onda corta reflejada de superficie / effective narrow band transmittance for shortwave radiation reflected from the surface.
\end{itemize}

\newpage
Estimación de Transmitancias

\begin{equation}
\tau_{\text{in},b} = C_1 \exp{
\left( 
  \frac{C_2 P}{K_t \cos{\theta_{hor}}} - \frac{C_3 W + C_4}{\cos{\theta_{\text{hor}}}}
\right) 
} + C_5
\label{eq:1}
\end{equation}

\begin{equation}
\tau_{\text{out},b} = C_1 \exp{
\left( 
  \frac{C_2 P}{K_t \cos{\eta}} - \frac{C_3 W + C_4}{\cos{\eta}}
\right) 
} + C_5
\label{eq:1}
\end{equation}

Dónde:
\begin{itemize}
  \item C1 - C5 : constantes dadas en el trabajo de Allen et al. (2007)
  \item Kt : el coeficiente de claridad que va de 0 a 1, Kt = 1 para aire limpio y Kt = 0.5 para aire turbio.
  \item $\theta$h : ángulo de incidencia solar = zenith angle (90-elevation angle)
\end{itemize}

\vspace{0.4cm}

\subsubsection{Radiación de onda corta entrante $R_{S\downarrow}$}

Incoming broad-band short-wave radiation, as direct and diffuse at the Earth’s surface $[W/m^2]$, represents the principal energy source for ET. Morse et al. (2000)
  
\begin{equation}
R_{S\downarrow} = \frac{G_{sc} \cos\theta_{rel} \tau_{sw}}{d^2}
\label{eq:1}
\end{equation}

Dónde:
\begin{itemize}
    \item $G_{sc}$ : Constante solar $[1367 \space W/m^2]$ 
    \item $\theta_{rel}$ : Ángulo de incidencia solar $[\text{radianes}]$
    \item $d^2$ : Cuadrado de la distancia relativa Tierra-Sol (OJO: También se menciona el inverso del cuadrado de la distancia relativa de tierra al sol)
    \item $\tau_{sw}$ : Es la transmitancia en un sentido con condiciones de claridad / Transmitancia de la banda ancha $[W/m^2 \space K]$
\end{itemize}
  
\vspace{0.4cm}
  
\subsubsection{Radiación de onda larga saliente $R_{L\uparrow}$}

Outgoing long-wave radiation, RL↑, emitted from the surface is
driven by surface temperature and surface emissivity. RL↑ is computed using the Stefan–Boltzmann equation Morse et al., 2000.

$$
R_{L\uparrow} = \epsilon_0  \sigma  T_{s}^{4}
$$

Dónde:
\begin{itemize}
    \item $R_{L\uparrow}$ : Radiación de onda larga saliente $[W/m^2]$
    \item $\epsilon_0$ : Emisividad superficial de banda ancha / Broad band surface emissivity
    \item $\sigma$ : Constante de Stefan-Boltzmann $(5.67*10^{-8}[W m^{-2}K^{-4}])$
    \item $T_s$ : Temperatura de brillo de superficie $[K]$
\end{itemize}

\vspace{0.4cm}

\subsubsection{Radiación de onda larga entrante $R_{L\downarrow}$}

Incoming Long-Wave Radiation

\begin{equation}
R_{L\downarrow} = \epsilon_a  \sigma  T_{a}^{4}
\label{eq:1}
\end{equation}

Dónde:
\begin{itemize}
    \item $R_{L\downarrow}$ : Radiación de onda larga entrante $[W/m^2]$
    \item $\epsilon_a$ : Effective atmospheric emissivity (dimensionless)
    \item $\sigma$ : Constante de Stefan-Boltzmann $(5.67*10^{-8}[W m^{-2}K^{-4}])$
    \item $T_a$ : Near-surface air temperature $[K]$
\end{itemize}
  
% \newpage % Leer seccion 1
\newpage

\subsection{Flujo de calor del suelo $G$}

Soil heat flux is the rate of heat storage in the soil and vegetation due to conduction. General METRIC applications compute G as a ratio $G/R_n$ using an empirical equation by Bastiaanssen (2000) representing values near midday

\begin{equation}
\frac{G}{R_n} = ( T_s - 273.15 ) (0.0038 + 0.0074 \alpha ) (1-0.98 \; \text{NDVI}^4) % \qquad \text{(26)}
\label{eq:1}
\end{equation}

Where:
\begin{itemize}
    \item $T_s$ : surface temperature (K)
    \item $\alpha$ : surface albedo
\end{itemize}

G is then calculated by multiplying $G/R_n$ by $R_n$. 

% \newpage  % Leer seccion 2
\chapter{FLUJO DE CALOR SENSIBLE} % $H$

Aquí va información sobre el flujo de calor sensible
  % Leer seccion 3
\chapter{REFERENCIAS}

Aquí van las referencias.

\end{document}
